\documentclass{article}
\begin{document}

\title{Analiza i badanie systemu widzenia maszynowego z inteligentnym czujnikiem wizyjnym}
\author{Marcin Ryzewski}
\date{\today}
\maketitle  

\section{Introduce}


\subsection{Objectives}

\subsection{Methodology}

\subsection{Work Characteristics}

\section{Computer Viosn}

Computer vision is a multidisciplinary field that focuses 
on enabling machines to interpret and understand visual 
information from the world, akin to human vision. 
It encompasses a variety of tasks, including image 
recognition, object detection, and scene understanding, 
which are achieved through the application of algorithms and 
models designed to process and analyze visual data. 
The core objective of computer vision is to automate the 
extraction of meaningful information from images and videos, 
facilitating applications in diverse domains such as 
autonomous driving, surveillance, and medical imaging.

The development of computer vision systems relies heavily 
on deep learning techniques, particularly convolutional 
neural networks (CNNs), which have demonstrated remarkable 
performance in various visual recognition tasks. These 
models learn hierarchical representations from raw pixel 
data, allowing them to identify patterns and features 
within images (Nguyen et al., 2015). However, the complexity 
of these models often leads to challenges in interpretability, 
making it difficult for users to understand how decisions 
are made based on visual inputs (Namatēvs et al., 2023). 
As a result, there is a growing emphasis on explainable 
AI within the field, aiming to enhance the transparency 
of computer vision systems and their decision-making 
processes (Escalante et al., 2017).

\section{AI Camera IMA Sony IMX500}

\end{document}